\documentclass[specialist, subf, href, colorlinks=true, 14pt, times, mtpro, final]{disser}
\usepackage [russian] {babel}
\usepackage [utf8] {inputenc}
\usepackage {amsmath}
\usepackage {amsthm}
\usepackage {amssymb}
\usepackage{wrapfig}
\usepackage{enumitem}
\usepackage{amsfonts}
\usepackage{textcomp}
\usepackage{graphicx}
\usepackage{float}
\usepackage{caption}
\usepackage{algorithm}
\usepackage{algpseudocode}
\usepackage{xcolor}
\usepackage{hyperref}

\begin{document}

\begin{center}
    Вопросы по курсу <<Численные методы>> \\ 4 курс, II поток
\end{center}

\noindent 1. Погрешность метода и вычислительная погрешность. Пример неустойчивого алгоритма.\\
\noindent 2. Алгебраическая интерполяция. Многочлен Лагранжа.\\
\noindent 3. Константа Лебега интерполяционного процесса для равноотстоящих узлов.\\
\noindent 4. Многочлены Чебышёва и их свойства.\\
\noindent 5. Интерполяционные сплайны. Конструкция и обоснование кубического сплайна.\\
\noindent 6. Понятие об аппроксимационных сплайнах.\\
\noindent 7. Наилучшее приближение в линейном нормированном пространстве.\\
\noindent 8. Наилучшее приближение в гильбертовом пространстве.\\
\noindent 9. Дискретное преобразование Фурье. Идея быстрого дискретного преобразования Фурье.\\
\noindent 10. Наилучшее равномерное приближение многочленами.\\
\noindent 11. Квадратурные формулы интерполяционного типа.\\
\noindent 12. Ортотональные многочлены и квадратуры Гаусса.
\noindent 13. Составные квадратурные формулы. Правило Рунге для оценки погрешности.\\
\noindent 14. Основные приёмы для вычисления нерегулярных интегралов.\\
\noindent 15. Метод прогонки для решения трёхдиагональных систем. Корректность и устойчивость метода прогонки.\\


\end{document}
