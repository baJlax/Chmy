\documentclass[specialist, subf, href, colorlinks=true, 14pt, times, mtpro, final]{disser}
\usepackage [russian] {babel}
\usepackage [utf8] {inputenc}
\usepackage {amsmath}
\usepackage {amsthm}
\usepackage {amssymb}
\usepackage{wrapfig}
\usepackage{enumitem}
\usepackage{amsfonts}
\usepackage{textcomp}
\usepackage{graphicx}
\usepackage{float}
\usepackage{caption}
\usepackage{algorithm}
\usepackage{algpseudocode}
\usepackage{xcolor}
\usepackage{hyperref}

\begin{document}
\begin{center}
    Вопросы по курсу <<Численные методы>> \\ 4 курс, II поток
\end{center}
{\small
\noindent 1. Погрешность метода и вычислительная погрешность. Пример неустойчивого алгоритма.\\
\noindent 2. Алгебраическая интерполяция. Многочлен Лагранжа.\\
\noindent 3. Константа Лебега интерполяционного процесса для равноотстоящих узлов.\\
\noindent 4. Многочлены Чебышёва и их свойства.\\
\noindent 5. Интерполяционные сплайны. Конструкция и обоснование кубического сплайна.\\
\noindent 6. Понятие об аппроксимационных сплайнах.\\
\noindent 7. Наилучшее приближение в линейном нормированном пространстве.\\
\noindent 8. Наилучшее приближение в гильбертовом пространстве.\\
\noindent 9. Дискретное преобразование Фурье. Идея быстрого дискретного преобразования Фурье.\\
\noindent 10. Наилучшее равномерное приближение многочленами.\\
\noindent 11. Квадратурные формулы интерполяционного типа.\\
\noindent 12. Ортотональные многочлены и квадратуры Гаусса.
\noindent 13. Составные квадратурные формулы. Правило Рунге для оценки погрешности.\\
\noindent 14. Основные приёмы для вычисления нерегулярных интегралов.\\
\noindent 15. Метод прогонки для решения трёхдиагональных систем. Корректность и устойчивость метода прогонки.\\
\noindent 16. Прямые методы решения систем линейных уравнений. Методы Гаусса и Холецкого.\\
\noindent 17. Прямые методы решения систем линейных уравнений. Методы отражений и вращений.\\
\noindent 18. Число обусловленности. Неравенства для ошибки и невязки.\\
\noindent 19. Метод простой итерации решения систем линейных уравнений.\\
\noindent 20. Оптимальный одношаговый итерационный метод.\\
\noindent 21. Оптимальный циклический итерационный метод.\\
\noindent 22. Обобщённый метод простой итерации.\\
\noindent 23. Методы Якоби и Гаусса -- Зейделя.\\
\noindent 24. Метод верхней релаксации.\\
\noindent 25. Метод наискорейшего градиентного спуска.\\
\noindent 26. Линейная задача наименьших квадратов. Метод нормального уравнения.\\
\noindent 27. Линейная задача наименьших квадратов. Методы QR-разложения и сингулярного разложения.\\
}
\end{document}
